\documentclass[type_judgement.tex]{subfiles}

\begin{document}

扱いたい構造は以下の全てを持つ圏
\begin{itemize}
    \item 自己双対コンパクト閉圏(Self--Dual Compact Closed Category)
    \item トレース付き有限双積(Traced Finite Biproduct)
    \item 半環圏(Rig Category)
    \item 逆圏(Inverse Category)
    \item 上限による豊穣化(Sup enrichment)
    \item 余代数モーダリティの余クライスリ圏(coKleisli Category of coAlgebra Modality)
    \item ダガーファイブレーション(Dagger Fibration)
    % \item 代数的コンパクト(Algebraically Compact)
\end{itemize}

\subsection{圏}
\begin{defn}[Category(圏)]
圏$\mathcal{C}$は以下の構造を備える
\begin{itemize}
    \item 対象(Object)の類$\Obj(\mathcal{C})$を持つ(各対象は$A,B,C$等の大文字で表す)
    \item $A,B\in\Obj(\mathcal{C})$に対する射(Morphism)の類$\Hom_\mathcal{C}(A,B)$を持つ(各射は単に$A \rightarrow B$で表すか,$f,g,h$等の小文字で表す)
        \begin{itemize}
            \item 射$f:A \rightarrow B$の$A$を始域(Domain),$B$を終域(Codomain)と呼び,それぞれ$\Dom(f)$,$\Cod(f)$で表す
        \end{itemize}
    \item 任意の対象$A$に対して恒等射$\id_A:A \rightarrow A$が存在する
    \item 任意の射$f:A \rightarrow B$及び$g:B \rightarrow C$に対して射の合成$g \circ f: A \rightarrow C$が存在する
    \item 任意の対象$A$,$B$及び射$f:A \rightarrow B$について,単位律$\id_B \circ f = f$ $f \circ \id_A = f$を満たす
    \item 任意の射$f:A \rightarrow B$, $g:B \rightarrow C$, $h:C \rightarrow D$に対して結合律$(h \circ g) \circ f = h \circ (g \circ f)$を満たす
\end{itemize}

\begin{defn}[Monomorphism(単射)]
射$f:A \rightarrow B$は,任意の射$g_1,g_2:C \rightarrow A$に対して$f \circ g_1 = f \circ g_2$ならば$g_1 = g_2$が成り立つとき,単射と呼ぶ.
\end{defn}

\begin{defn}[Epimorphism(全射)]
射$f:A \rightarrow B$は,任意の射$g_1,g_2:B \rightarrow C$に対して$g_1 \circ f = g_2 \circ f$ならば$g_1 = g_2$が成り立つとき,全射と呼ぶ.
\end{defn}

\begin{defn}[Bimorphism(双射)]
単射かつ全射である射を全単射もしくは双射と呼ぶ.
\end{defn}

\begin{defn}[Isomorphism(同型射)]
射$f:A \rightarrow B$が同型射(Isomorphism)であるとは,$g \circ f = \id_A$かつ$f \circ g = \id_B$を満たす$g:B \rightarrow A$が存在することをいう.このとき$g$は$f$の逆射(Inverse morphism)と呼び$f^{-1}$で表す.
二つの等式のうち,左のみ満たす場合の射$g$を$f$の引き込み(Retraction)と呼び,右のみ満たす場合の$g$を$f$の断面(Section)と呼ぶ.
射が同型射であることを強調する場合,$f:A \xrightarrow{\cong} B$と表す.
圏の対象$A,B$の間に同型射が存在するとき,$A$と$B$は同型(Isomorphic)であるといい,$A \cong B$で表す.
\end{defn}

任意の同型射は双射だが,双射は必ずしも同型射ではないことに留意すること.

\begin{defn}[Functor(関手)]
$\mathcal{C}$と$\mathcal{D}$を圏とする.関手$\mathbf{F}:\mathcal{C} \rightarrow \mathcal{D}$は,$\mathcal{C}$の対象を$\mathcal{D}$の対象へ写す関数$\Obj(\mathcal{C}) \rightarrow \Obj(\mathcal{D})$と,$\mathcal{C}$の射を$\mathcal{D}$の射に写す関数$\Hom_\mathcal{C}(A,B) \rightarrow \Hom_\mathcal{D}(\mathbf{F}A,\mathbf{F}B)$であり,
$\mathbf{F}(g \circ f) = \mathbf{F}(g) \circ \mathbf{F}(f)$と$\mathbf{F}(\id_A) = \id_{(\mathbf{F}A)}$を満たす.
\end{defn}

% \begin{defn}[Fully faithful ]

% \end{defn}

\begin{defn}[Natural Transformation(自然変換)]
$\mathcal{C}$と$\mathcal{D}$を圏とし,$\mathbf{F},\mathbf{G}:\mathcal{C} \rightarrow \mathcal{D}$を関手とする.自然変換$\tau:\mathbf{F} \Rightarrow \mathbf{G}$は,射の族$\tau_A:\mathbf{F}A \rightarrow \mathbf{G}A$から成り,圏$\mathcal{C}$の任意の射$f:A \rightarrow B$について,以下の図式が可換(矢印をどの順番に通っても,射の合成に関して等式が成り立つこと)になる.
\begin{center}
\begin{tikzcd}
  \mathbf{F}A \ar[r, "\tau_A"] \ar[d, "\mathbf{F}f"'] & \mathbf{G}A \ar[d, "\mathbf{G}f"] \\
  \mathbf{F}B \ar[r, "\tau_B"] & \mathbf{G}B
\end{tikzcd}
\end{center}
任意の対象$A\in\Obj(\mathcal{C})$について射$\tau_A$が$\mathcal{D}$の同型射であるとき,$\tau$は自然同型(Natural Isomorphism)であるという.
\end{defn}

\begin{defn}[Categorical Equivalence(圏同値)]
圏$\mathcal{C}$と$\mathcal{D}$は,関手$\mathbf{F}:\mathcal{C} \rightarrow \mathcal{D}$と$\mathbf{G}:\mathcal{D} \rightarrow \mathcal{C}$が存在して$\mathbf{Id}_\mathcal{C} \cong \mathbf{G}\circ\mathbf{F}$かつ$\mathbf{Id}_\mathcal{D} \cong \mathbf{F}\circ\mathbf{G}$であるとき,同値であるという($\mathbf{Id}_\mathcal{C}$と$\mathbf{Id}_\mathcal{D}$は各圏の恒等関手).
圏同値は$\mathcal{C} \simeq \mathcal{D}$で表す.
\end{defn}

\begin{defn}[Opposite Category(反対圏)]
圏$\mathcal{C}$の反対圏$\mathcal{C}^{op}$とは,$\Obj(\mathcal{C}^{op}) = \Obj(\mathcal{C})$かつ$\Hom_{\mathcal{C}^{op}}(A,B) = \Hom_\mathcal{C}(B,A)$である圏である.
\end{defn}

\begin{defn}[Adjunction(随伴)\cite{mac_lane_categories_1978}]
  $\mathcal{C}$と$\mathcal{D}$を圏とする.関手$\mathbf{L}:\mathcal{C} \rightarrow \mathcal{D}$と$\mathbf{R}:\mathcal{D} \rightarrow \mathcal{C}$が随伴(Adjunction)であるとは,
  対象$A\in\Obj(\mathcal{C})$と$B\in\Obj(\mathcal{D})$について自然となる以下の自然同型が成り立つときであり,$\mathbf{L}\dashv\mathbf{R}$であらわす.
  \begin{equation*}
    \Hom_\mathcal{D}(\mathbf{L}A, B) \cong \Hom_\mathcal{C}(A, \mathbf{R}B)
  \end{equation*}
  \begin{center}
    \begin{tikzcd}
      \mathcal{C}
      \arrow[r, "\mathbf{L}"{name=L}, bend left=25] &
      \mathcal{D}
      \arrow[l, "\mathbf{R}"{name=R}, bend left=25]
      %--- Adjunction Symbol
      \arrow[phantom, from=L, to=R, "\dashv" rotate=-90]
    \end{tikzcd}
  \end{center}
\end{defn}

\subsection{モノイダル圏}
\end{defn}
\begin{defn}[Monoidal Category(モノイダル圏)\cite{selinger09}]
モノイダル圏$\mathcal{C}$は以下の構造を備える
\begin{itemize}
    \item 双関手$(-)\otimes(-):\mathcal{C}\times\mathcal{C} \rightarrow \mathcal{C}$(モノイダル積と呼ばれる)
    \item 単位対象$\mathrm{I}\in\Obj(\mathcal{C})$
    \item (自然同型)結合子$\alpha_{A,B,C}:(A \otimes B)\otimes C \xrightarrow{\cong} A \otimes (B \otimes C)$
    \item (自然同型)左単位子$\lambda_A:\mathrm{I} \otimes A \xrightarrow{\cong} A$
    \item (自然同型)右単位子$\rho_A:A \otimes \mathrm{I} \xrightarrow{\cong} A$
    \item 二つの図式(五角等式,三角等式)が可換となる
\end{itemize}
\begin{center}
\begin{tikzpicture}
    \node[
        regular polygon,
        regular polygon sides=5,
        minimum width=50mm,
        yscale=.75,
        xscale=1.75
    ] (PG) {}
      (PG.corner 1) node (PG1) {$(A \otimes B)\otimes(C \otimes D)$}
      (PG.corner 2) node (PG2) {$((A \otimes B) \otimes C )\otimes D$}
      (PG.corner 3) node (PG3) {$(A\otimes(B \otimes C))\otimes D$}
      (PG.corner 4) node (PG4) {$A\otimes((B\otimes C)\otimes D)$}
      (PG.corner 5) node (PG5) {$A\otimes(B\otimes(C\otimes D))$}
    ;
    \draw[->] (PG1) -- (PG5) node[midway, above right] {$\alpha_{A, B, C \otimes D}$};
    \draw[->] (PG2) -- (PG1) node[midway, above left] {$\alpha_{A \otimes B, C, D}$};
    \draw[->] (PG2) -- (PG3) node[midway, left] {$\alpha_{A,B,C} \otimes \id_D$};
    \draw[->] (PG3) -- (PG4) node[midway, below] {$\alpha_{A, B \otimes C, D}$};
    \draw[->] (PG4) -- (PG5) node[midway, right] {$\id_A \otimes \alpha_{B,C,D}$};
\end{tikzpicture}
\end{center}
\begin{center}
\begin{tikzpicture}
    \node[
        regular polygon,
        regular polygon sides=3,
        shape border rotate=180,
        minimum width=40mm,
        yscale=.5,
        xscale=1.25
    ] (PG) {}
      (PG.corner 1) node (PG1) {$A \otimes B$}
      (PG.corner 2) node (PG2) {$A \otimes (\mathrm{I} \otimes B)$}
      (PG.corner 3) node (PG3) {$(A \otimes \mathrm{I}) \otimes B$}
    ;
    \draw[->] (PG2) -- (PG1) node[midway, below right] {$\id_A \otimes \lambda_B$};
    \draw[->] (PG3) -- (PG1) node[midway, below left] {$\rho_A \otimes \id_B$};
    \draw[->] (PG3) -- (PG2) node[midway, above] {$\alpha_{A,\mathrm{I},B}$};
\end{tikzpicture}
\end{center}
\end{defn}

\begin{defn}[Symmetric Monoidal Category(対称モノイダル圏)\cite{selinger09}]
対称モノイダル圏$\mathcal{C}$はモノイダル圏であり,以下の構造を備える.
\begin{itemize}
    \item (自然同型)対称子$\sigma_{A,B}:A \otimes B \xrightarrow{\cong} B \otimes A$
    \item 等式$\sigma_{A,B} = \sigma_{B,A}^{-1}$を満たす
    \item 六角等式が可換となる
\end{itemize}
\begin{center}
\begin{tikzpicture}
    \node[
        regular polygon,
        regular polygon sides=6,
        minimum width=50mm,
        yscale=.75,
        xscale=1.75
    ] (PG) {}
      (PG.corner 1) node (PG1) {$B \otimes (A \otimes C)$}
      (PG.corner 2) node (PG2) {$(B \otimes A) \otimes C$}
      (PG.corner 3) node (PG3) {$(A \otimes B) \otimes C$}
      (PG.corner 4) node (PG4) {$A \otimes (B \otimes C)$}
      (PG.corner 5) node (PG5) {$(B \otimes C) \otimes A$}
      (PG.corner 6) node (PG6) {$B \otimes (C \otimes A)$}
    ;
    \draw[->] (PG1) -- (PG6) node[midway, above right] {$\id_B \otimes \sigma_{A,C}$};
    \draw[->] (PG2) -- (PG1) node[midway, above] {$\alpha_{B,A,C}$};
    \draw[->] (PG3) -- (PG2) node[midway, above left] {$\sigma_{A,B} \otimes \id_C$};
    \draw[->] (PG3) -- (PG4) node[midway, below left] {$\alpha_{A,B,C}$};
    \draw[->] (PG4) -- (PG5) node[midway, below] {$\sigma_{A, B \otimes C}$};
    \draw[->] (PG5) -- (PG6) node[midway, below right] {$\alpha_{B,C,A}$};
\end{tikzpicture}
\end{center}
\end{defn}

\begin{defn}[Symmetric Monoidal Closed Category(対称モノイダル閉圏)\cite{barr91}]
対称モノイダル閉圏$\mathcal{C}$は対象モノイダル圏であり,以下の構造を備える.
\begin{itemize}
    \item 双関手$(-)\multimap(-):\mathcal{C}^{op}\times\mathcal{C} \rightarrow \mathcal{C}$(内部ホムと呼ばれる)
    \item 任意の対象$A\in\Obj(\mathcal{C})$について,モノイダル積$\otimes$と内部ホム$\multimap$の随伴$((-)\otimes{A} \dashv A \multimap(-)):\mathcal{C} \rightarrow \mathcal{C}$が存在する
\end{itemize}
\end{defn}

\begin{prop}
\label{prop:mc_adj}
対称モノイダル閉圏$\mathcal{C}$は任意の$A,B,C\in\Obj(\mathcal{C})$について、同型$(A \otimes B) \multimap C \cong A \multimap (B \multimap C)$を持つ.
\end{prop}
\begin{proof}
モノイダル積$\otimes$と内部ホム$\multimap$の随伴から自然同型
\begin{equation*}
    \Hom_\mathcal{C}(A \otimes B, C) \cong \Hom_\mathcal{C}(A, B \multimap C)
\end{equation*}
が得られる.
任意の$X\in\Obj(\mathcal{C})$について,自然同型の合成により,
\begin{align*}
    \Hom_\mathcal{C}(X, (A \otimes B) \multimap C) &\xrightarrow{\cong} \Hom_\mathcal{C}(X \otimes (A \otimes B), C) \\
     &\xrightarrow{\cong} \Hom_\mathcal{C}((X \otimes A) \otimes B, C) \\
     &\xrightarrow{\cong} \Hom_\mathcal{C}(X \otimes A, B \multimap C) \\
     &\xrightarrow{\cong} \Hom_\mathcal{C}(X, A \multimap (B \multimap C))
\end{align*}
を得る.
米田の補題より,同型$(A \otimes B) \multimap C \cong A \multimap (B \multimap C)$が存在する.
\end{proof}

\begin{prop}
\label{prop:unit}
対称モノイド閉圏$\mathcal{C}$は任意の$X\in\Obj(\mathcal{C})$について、同型$\mathrm{I} \multimap A \cong A$を持つ.
\end{prop}
\begin{proof}
任意の$X\in\Obj(\mathcal{C})$について、自然同型の合成により,
\begin{align*}
    \Hom_\mathcal{C}(X, \mathrm{I} \multimap A) &\xrightarrow{\cong} \Hom_\mathcal{C}(X \otimes \mathrm{I}, A) \\
    &\xrightarrow{\cong} \Hom_\mathcal{C}(X, A)
\end{align*}
を得る.米田の補題より、同型$\mathrm{I} \multimap A \cong A$が存在する.
\end{proof}

\begin{prop}
\label{prop:mc_c}
対称モノイド閉圏$\mathcal{C}$は任意の$A,B,C\in\Obj(\mathcal{C})$について,以下の3つの射を持つ.
\begin{align*}
    \eta_{A}'    &: \mathrm{I} \rightarrow A \multimap A \\
    \eval_{A,B}   &: (A \multimap B) \otimes A \rightarrow B \\
    \comp_{A,B,C} &: (A \multimap C) \otimes (B \multimap C) \rightarrow A \multimap C
\end{align*}
\end{prop}
\begin{proof}
恒等射$\id_A$より,射の同型
\begin{align*}
    \Hom_\mathcal{C}(A,A) &\xrightarrow{\cong} \Hom_\mathcal{C}(\mathrm{I} \otimes A, A) \\
    &\xrightarrow{\cong} \Hom_\mathcal{C}(\mathrm{I}, A \multimap A)
\end{align*}
を得るので,射$\eta_{A}$は存在する.
次に,恒等射$\id_{A \multimap B}$より,射の同型
\begin{align*}
    \Hom_\mathcal{C}(A \multimap B, A \multimap B) &\xrightarrow{\cong} \Hom_\mathcal{C}((A \multimap B) \otimes A, B)
\end{align*}
を得るので,射$\eval_{A,B}$は存在する.
次に,合成射$\eval_{B,C}\circ(\id_{B\multimap{}C}\otimes \eval_{A,B})$より,射の同型
\begin{align*}
    \Hom_\mathcal{C}((B \multimap C)\otimes((A \multimap B)\otimes A), C) &\xrightarrow{\cong} \Hom_\mathcal{C}( ((B \multimap C)\otimes(A \multimap B)) \otimes A, C) \\
    &\xrightarrow{\cong} \Hom_\mathcal{C}((B \multimap C) \otimes (A \multimap B), A \multimap C)
\end{align*}
を得るので,射$\comp_{A,B,C}$は存在する.
\end{proof}

命題\ref{prop:mc_adj},命題\ref{prop:unit},命題\ref{prop:mc_c}は,対称子を使用せず証明しており,モノイド閉圏でも成り立つ.

命題\ref{prop:mc_c}で用いた射$\eval_{A,B}$より,射の同型から
\begin{align*}
    \Hom_\mathcal{C}((A\multimap B) \otimes A, B) &\xrightarrow{\cong} \Hom_\mathcal{C}(A \otimes (A \multimap B), B) \\
    &\xrightarrow{\cong} \Hom_\mathcal{C}(A, (A \multimap B) \multimap B)
\end{align*}
を得る.

% 定義に漏れあり
% \begin{defn}[Closed Category(閉圏)]
% 閉圏$\mathcal{C}$は以下の構造を備える.
% \begin{itemize}
%     \item 双関手$(-)\multimap(-):\mathcal{C}^{op} \times \mathcal{C} \rightarrow \mathcal{C}$
%     \item 単位対象$I\in\Obj(\mathcal{C})$
%     \item (自然同型)$o_A:A \cong I \multimap A$
%     \item (dinatural変換)$\vartheta_A:I \rightarrow A \multimap A$
%     \item ($B,C$に対して自然,$A$ に対してdinaturalな変換)$L_{A,B,C}:B \multimap C \rightarrow (A \multimap B) \multimap (A \multimap C)$
%     \item 以下の図式が全て可換
%         \begin{itemize}
%             \item 
%         \end{itemize}
% \end{itemize}
% \end{defn}

\begin{defn}[$\star\mathchar`-$autonomous Category($\star\mathchar`-$自立圏)\cite{barr91}]
\label{def:star-auto}
$\star\mathchar`-$自立圏$\mathcal{C}$は対称モノイド閉圏であり,以下の構造を備える.
\begin{itemize}
    \item 双対化対象$\bot\in\Obj(\mathcal{C})$
    \item 双対化関手$(-)^\star:=(-)\multimap \bot:\mathcal{C}^{op} \rightarrow \mathcal{C}$
    \item (自然同型)二重双対$A \xrightarrow{\cong} A^{\star\star}$
\end{itemize}
\end{defn}

双対化関手$(-)^\star$から射の同型
\begin{align*}
    \Hom_\mathcal{C}(A \otimes B, C^\star) &= \Hom_\mathcal{C}(A \otimes B, C \multimap \bot) \\
    &\xrightarrow{\cong} \Hom_\mathcal{C}((A \otimes B) \otimes C, \bot) \\
    &\xrightarrow{\cong} \Hom_\mathcal{C}(A \otimes (B \otimes C),  \bot) \\
    &\xrightarrow{\cong} \Hom_\mathcal{C}(A, (B \otimes C) \multimap \bot) \\
    &= \Hom_\mathcal{C}(A, (B \otimes C)^\star)
\end{align*}
を得る.

二重双対により,以下の同型
\begin{align*}
    A \otimes B &\cong (A \multimap B^\star)^\star  &  A \multimap B &\cong (A \otimes B^\star)^\star  &  A \multimap B &\cong B^\star \multimap A^\star
\end{align*}
を得る.
\begin{proof}
\begin{align*}
&\begin{aligned}
    A \otimes B &\xrightarrow{\cong} (A \otimes B)^{\star\star} \\
    &= ((A \otimes B) \multimap \bot)^\star \\
    &\xrightarrow{\cong} (A \multimap (B \multimap \bot))^\star \\
    &= (A \multimap B^\star)^\star \\
    & \\
    &
\end{aligned}
&
&\begin{aligned}
    A \multimap B &\xrightarrow{\cong} A \multimap B^{\star\star} \\
    &= A \multimap (B^\star \multimap \bot) \\
    &\xrightarrow{\cong} (A \otimes B^\star) \multimap \bot \\
    &= (A \otimes B^\star)^\star \\
    & \\
    &
\end{aligned}
&
&\begin{aligned}
    A \multimap B &\xrightarrow{\cong} A \multimap B^{\star\star} \\
    &= A \multimap (B^\star \multimap \bot) \\
    &\xrightarrow{\cong} (A \otimes B^\star) \multimap \bot \\
    &\xrightarrow{\cong} (B^\star \otimes A) \multimap \bot \\
    &\xrightarrow{\cong} B^\star \multimap (A \multimap \bot) \\
    &= B^\star \multimap A^\star
\end{aligned}
\end{align*}
\end{proof}

同様に,自然同型の合成により$\Hom_\mathcal{C}(A,B) \cong \Hom_\mathcal{C}(B^\star, A^\star)$を得る.
また,命題\ref{prop:unit}より,同型$\mathrm{I}^\star \cong \bot$を得る.

\begin{defn}[Compact Closed Category(コンパクト閉圏)\cite{abramsky09}\cite{kelly80}]
コンパクト閉圏$\mathcal{C}$は$\star\mathchar`-$自立圏であり,以下の構造を備える.
\begin{itemize}
    \item (自然同型)自己双対$\varsigma_{A,B}:(A \otimes B)^\star \xrightarrow{\cong} A^\star \otimes B^\star$
    \item (自然同型)自己双対$\varsigma_\mathrm{I}:\mathrm{I}^\star \xrightarrow{\cong} \mathrm{I}$
\end{itemize}
\end{defn}

コンパクト閉圏の定義から同型を得る.
\begin{align*}
    A \multimap B &\xrightarrow{\cong} (A \otimes B^\star)^\star \\
    &\xrightarrow{\cong} (A^\star \otimes B^{\star\star}) \\
    &\xrightarrow{\cong} A^\star \otimes B
\end{align*}
また,コンパクト閉圏$\mathcal{C}$の任意の対象$A,B,C$について,以下の射の同型を得る.
\begin{equation*}
    \Hom_\mathcal{C}(A, B^\star \otimes C) \cong \Hom_\mathcal{C}(A, B \multimap C) \cong \Hom_\mathcal{C}(A \otimes B, C)
\end{equation*}

コンパクト閉圏$\mathcal{C}$の任意の対象$A\in\Obj(\mathcal{C})$の恒等射$\id_A$について,二つの射の同型
\begin{align*}
&\begin{aligned}
    \Hom_\mathcal{C}(A, A) &\xrightarrow{\cong} \Hom_\mathcal{C}(\mathrm{I} \otimes A, A) \\
    &\xrightarrow{\cong} \Hom_\mathcal{C}(\mathrm{I}, A \multimap A) \\
    &\xrightarrow{\cong} \Hom_\mathcal{C}(\mathrm{I}, A^\star \otimes A) \\
    &
\end{aligned}
&
&\begin{aligned}
    \Hom_\mathcal{C}(A, A) &\xrightarrow{\cong} \Hom_\mathcal{C}(A, \mathrm{I} \multimap A) \\
    &\xrightarrow{\cong} \Hom_\mathcal{C}(A, A^\star \multimap \mathrm{I}^\star) \\
    &\xrightarrow{\cong} \Hom_\mathcal{C}(A, A^\star \multimap \mathrm{I}) \\
    &\xrightarrow{\cong} \Hom_\mathcal{C}(A \otimes A^\star, \mathrm{I})
\end{aligned}
\end{align*}
から,$\eta_A:\mathrm{I} \rightarrow A^\star \otimes A$と$\epsilon_A:A \otimes A^\star \rightarrow \mathrm{I}$を得る.

コンパクト閉圏$\mathcal{C}$の任意の対象$A$について,$\eta_A$から射$\decomp_{A,B,C}: A \multimap C \rightarrow (A \multimap B) \otimes (B \multimap C)$を以下の射の合成として得る
\begin{align*}
    A \multimap C &\xrightarrow{\cong} (A^\star \otimes C) \\
                  &\xrightarrow{\rho_{A^\star \otimes C}} (A^\star \otimes C) \otimes \mathrm{I} \\
                  &\xrightarrow{\id_{A^\star \otimes C} \otimes \eta_B} (A^\star \otimes C) \otimes (B^\star \otimes B) \\
                  &\xrightarrow{\id_{A^\star \otimes C} \otimes \sigma_{B^\star,B}} (A^\star \otimes C) \otimes (B \otimes B^\star) \\
                  &\xrightarrow{\alpha_{A^\star,C,B \otimes B^\star}} A^\star \otimes (C \otimes (B \otimes B^\star)) \\
                  &\xrightarrow{\id_{A^\star} \otimes \sigma_{C,B \otimes B^\star}} A^\star \otimes ((B \otimes B^\star) \otimes C) \\
                  &\xrightarrow{\alpha^{-1}_{A^\star, B \otimes B^\star, C}} (A^\star \otimes (B \otimes B^\star)) \otimes C \\
                  &\xrightarrow{\alpha^{-1}_{A^\star, B, B^\star} \otimes \id_C} ((A^\star \otimes B) \otimes B^\star) \otimes C \\
                  &\xrightarrow{\alpha_{A^\star \otimes B, B^\star, C}} (A^\star \otimes B) \otimes (B^\star \otimes C) \\
                  &\xrightarrow{\cong} (A \multimap B) \otimes (B \multimap C)
\end{align*}

\subsection{ダガー圏}
\begin{defn}[$\dagger\mathchar`-$Category($\dagger\mathchar`-$圏)\cite{selinger09}]
$\dagger\mathchar`-$圏$\mathcal{C}$は,以下の構造を備える.
\begin{itemize}
    \item 関手$\dagger:\mathcal{C}^{op} \rightarrow \mathcal{C}$
    \item $\dagger$は対象に対して恒等
    \item 対合的$\dagger\circ\dagger = \mathbf{Id}_\mathcal{C}$
\end{itemize}
\end{defn}
$\dagger\mathchar`-$圏の定義から,任意の射$:A \rightarrow B, g:B \rightarrow C$について,以下の等式が導かれる.
\begin{alignat*}{2}
    \id_A^\dagger       &= \id_A                      &:A &\rightarrow A \\
    (g\circ f)^\dagger &= f^\dagger \circ g^\dagger &:C &\rightarrow A \\
    f^{\dagger\dagger} &= f                         &:A &\rightarrow B,
\end{alignat*}

$f^\dagger=f^{-1}$となる$f$をユニタリーと呼び,$f^\dagger=f$となる$f$を自己随伴と呼ぶ.
% $^\dagger\mathchar`-$関手$\mathbf{F}$は$\dagger\mathchar`-$圏間の関手であり,任意の射$f$について,$\mathbf{F}(f^\dagger)=(\mathbf{F}f)^\dagger$を満たす.

\begin{defn}[$\dagger\mathchar`-$Functor($\dagger\mathchar`-$関手)]
  {\bf ToDo}
\end{defn}

\begin{defn}[$\dagger\mathchar`-$Symmetric Monoidal Category($\dagger\mathchar`-$対称モノイド圏)\cite{selinger09}\cite{abramsky09}]
$\dagger\mathchar`-$対称モノイド圏$\mathcal{C}$は,対称モノイド圏かつ$\dagger\mathchar`-$圏であり,以下の構造を備える.
\begin{itemize}
    \item 任意の$A,B,C\in\Obj(\mathcal{C})$について,$\alpha_{A,B,C},\lambda_{A},\rho_{A},\sigma_{A,B}$がユニタリ
    \item 関手$\dagger$が厳密モノイド関手であり,任意の射$f,g$について,$(f\otimes g)^\dagger = f^\dagger \otimes g^\dagger$
\end{itemize}
\end{defn}

\begin{defn}[$\dagger\mathchar`-$Comact Category($\dagger\mathchar`-$コンパクト圏)\cite{abramsky09}]
$\dagger\mathchar`-$コンパクト圏$\mathcal{C}$は,コンパクト閉圏かつ$\dagger\mathchar`-$対称モノイド圏であり,以下の構造を備える.
\begin{itemize}
    \item 任意の$A\in\Obj(\mathcal{C})$について,以下の図式が可換\\
    \begin{center}
        \begin{tikzcd}
            \mathrm{I}
            \ar[r, "\eta_A"]
            \ar[dr, "\epsilon_A^\dagger"']
            & A^\star \otimes A
            \ar[d, "\sigma_{A^\star, A}"] \\
            & A \otimes A^\star
        \end{tikzcd}
    \end{center}
\end{itemize}
\end{defn}
定義から$\epsilon_A$は$\epsilon_A=\eta_A^\dagger\circ\sigma_{A,A^\star}$として定義可能となる.逆に$\epsilon_A$から$\eta_A$を定義することもできる.

$\dagger\mathchar`-$コンパクト圏$\mathcal{C}$は,任意の対象$A,B$の内部ホム$A \multimap B$について,以下の図式を可換にする射$\mathrm{dagger}_{A,B}:A \multimap B \rightarrow B \multimap A$を持つ.
\begin{center}
    \begin{tikzcd}[column sep=3cm]
        \Hom(\mathrm{I},A\multimap{}B) \ar[r, "{\Hom(\mathrm{I},\mathrm{dagger}_{A,B})}"] \ar[d, "\cong"'] & \Hom(\mathrm{I},B\multimap{}A) \\
        \Hom(A,B) \ar[r, "(-)^\dagger"] & \Hom(B,A) \ar[u, "\cong"]
    \end{tikzcd}
\end{center}

\subsection{豊饒圏}
\begin{defn}[$\mathcal{V}$--圏($\mathcal{V}$--category)]\cite{kelly_basic_nodate}
$\mathcal{V}$をモノイダル圏とする.$\mathbb{A}$が$\mathcal{V}$で豊穣化された圏(あるいは単に$\mathcal{V}$--圏)とは以下の要素からなる.
\begin{itemize}
  \item $\mathbb{A}$の対象$\Obj(\mathbb{A})$
  \item 任意の対象$A,B \in \Obj(\mathbb{A})$に対してホム対象$\mathbb{A}(A,B) \in \Obj(\mathcal{V})$
  \item 任意の対象$A \in \Obj(\mathbb{A})$に対する$\mathcal{V}$の射$u_{A}:I \rightarrow \mathbb{A}(A,A)$
  \item 任意の対象$A,B,C \in \Obj(\mathbb{A})$に対する$\mathcal{V}$の射$c_{A,B,C}:\mathbb{A}(B,C) \otimes \mathbb{A}(A,B) \rightarrow \mathbb{A}(A,C)$
  \item 任意の対象$A,B,C,D \in \Obj(\mathbb{A})$に対して以下の可換図式を満たす
\end{itemize}
\begin{center}
  \begin{tikzpicture}
      \node[
          regular polygon,
          regular polygon sides=5,
          minimum width=50mm,
          yscale=.75,
          xscale=2.75
      ] (PG) {}
        (PG.corner 1) node (PG1) {$\mathbb{A}(A,D)$}
        (PG.corner 2) node (PG2) {$\mathbb{A}(C,D)\otimes\mathbb{A}(A,C)$}
        (PG.corner 3) node (PG3) {$\mathbb{A}(C,D)\otimes(\mathbb{A}(B,C)\otimes\mathbb{A}(A,B))$}
        (PG.corner 4) node (PG4) {$(\mathbb{A}(C,D)\otimes\mathbb{A}(B,C))\otimes\mathbb{A}(A,B)$}
        (PG.corner 5) node (PG5) {$\mathbb{A}(B,D)\otimes\mathbb{A}(A,B)$}
      ;
      \draw[->] (PG5) -- (PG1) node[midway, above right] {$c_{A,B,D}$};
      \draw[->] (PG2) -- (PG1) node[midway, above left] {$c_{A,C,D}$};
      \draw[->] (PG3) -- (PG2) node[midway, left] {$u_{\mathbb{A}(C,D)}\otimes c_{A,B,C}$};
      \draw[->] (PG4) -- (PG3) node[midway, below] {$\alpha_{\mathbb{A}(C,D),\mathbb{A}(B,C),\mathbb{A}(A,B)}$};
      \draw[->] (PG4) -- (PG5) node[midway, right] {$c_{B,C,D}\otimes u_{\mathbb{A}(A,B)}$};
  \end{tikzpicture}
  \end{center}
  \begin{center}
  \begin{tikzcd}[row sep=huge]
    \mathbb{A}(B,B)\otimes\mathbb{A}(A,B) \arrow[r,"c_{A,B,B}"] &
    \mathbb{A}(A,B) &
    \mathbb{A}(A,B)\otimes\mathbb{A}(A,A) \arrow[l,"c_{A,A,B}"']
    \\
    I\otimes\mathbb{A}(A,B) \arrow[u,"u_B\otimes\id_{\mathbb{A}(A,B)}"] \arrow[ur,"\lambda_{\mathbb{A}(A,B)}"'] &
    &
    \mathbb{A}(A,B)\otimes I \arrow[u,"\id_{\mathbb{A}(A,B)}\otimes u_A"'] \arrow[ul,"\rho_{\mathbb{A}(A,B)}"]
    \end{tikzcd}
  \end{center}
\end{defn}

{\bf ToDo} 必要なら$\mathcal{V}$--関手,$\mathcal{V}$--自然変換も定義する.

\subsection{有限双積}
\begin{defn}[Finite Product(有限積)\cite{selinger09}]
圏$\mathcal{C}$における対象$A,B$に対して有限積$A \times B$は,以下の構造を備える.
\begin{itemize}
    \item 射$\pi_1:A \times B \rightarrow A$と$\pi_2:A \times B \rightarrow B$
    \item 任意の対象$C$と射の組$f:C \rightarrow A, g:C \rightarrow B$に対して一意な射$h:C \rightarrow A \times B$が存在して以下の図式が可換
    \begin{center}
        \begin{tikzcd}
            & C
            \ar[dl, "f"']
            \ar[d, dashed, "h"]
            \ar[dr, "g"] \\
            A
            & A \times B
            \ar[l, "\pi_1"]
            \ar[r, "\pi_2"']
            & B
        \end{tikzcd}
    \end{center}
    射$h$は$\langle f,g \rangle$と表記されることが多い.
    \item (終対象)任意の対象$C$に対して一意な射$h:C \rightarrow \mathrm{I}$が存在する対象$\mathrm{I}$
\end{itemize}

\end{defn}
有限積からなる圏(デカルト圏)は,以下の構造と満たすべきいくつかの公理\cite{selinger09}を持つ対称モノイド圏と同値である.
\begin{itemize}
    \item (自然な射の族)複製$\Delta_A:A \rightarrow A \otimes A$
    \item (自然な射の族)削除$\Diamond_A:A \rightarrow \mathrm{I}$
\end{itemize}

\begin{defn}[Finite Coproduct(有限余積)\cite{selinger09}]]
圏$\mathcal{C}$における対象$A,B$に対して有限余積$A + B$は,以下の構造を備える.
\begin{itemize}
    \item 射$\iota_1:A \rightarrow A \oplus B$と$\iota_2:B \rightarrow A \oplus B$
    \item 任意の対象$C$と射の組$f:A \rightarrow C, g:B \rightarrow C$に対して一意な射$h:A + B \rightarrow C$が存在して以下の図式が可換
    \begin{center}
        \begin{tikzcd}
            & C \\
            A
            \ar[r, "\iota_1"']
            \ar[ur, "f"]
            & A + B
            \ar[u, dashed, "h"']
            & B
            \ar[l, "\iota_2"]
            \ar[ul, "g"']
        \end{tikzcd}
    \end{center}
        射$h$は$[ f,g ]$と表記されることが多い.
    \item (始対象)任意の対象$C$に対して一意な射$h:\mathrm{I} \rightarrow C$が存在する対象$\mathrm{I}$
\end{itemize}
\end{defn}

有限余積からなる圏(余デカルト圏)は,以下の構造と満たすべきいくつかの公理\cite{selinger09}を持つ対称モノイド圏と同値である.
\begin{itemize}
    \item (自然な射の族)合併$\nabla_A:A \oplus A \rightarrow A$
    \item (自然な射の族)追加$\square_A:\mathrm{I} \rightarrow A$
\end{itemize}

\begin{defn}[Finite Biproduct(有限双積)\cite{selinger09}]
圏$\mathcal{C}$における対象$A_1,A_2$に対して有限双積$A_1 \oplus A_2$は,以下の構造を備える.
\begin{itemize}
    \item (零対象)任意の対象$A_1,A_2$に対して一意な零射$0_{A_1,A_2}:A_1 \rightarrow O \rightarrow A_2$が存在する対象$O$
    \item 射$\pi_1:A_1 \oplus A_2 \rightarrow A_1$と$\pi_2:A_1 \oplus A_2 \rightarrow A_2$
    \item $\pi_1,\pi_2$に関して有限積
    \item 射$\iota_1:A_1 \rightarrow A_1 \oplus A_2$と$\iota_2:A_2 \rightarrow A_1 \oplus A_2$
    \item$\iota_1,\iota_2$に関して有限余積
    \item
    $
    \delta_{ij} = \pi_i \circ \iota_j =
        \begin{cases}
            \id_A        & \text{if}\quad i = j \\
            0_{A_j,A_i} & \text{if}\quad i \neq j
        \end{cases}
    $
\end{itemize}
ここで零射$0_{A_1,A_2}$は任意の対象$X,Y,Z\in \Obj(\mathcal{C})$および任意の射$f:Y \rightarrow Z, g:X \rightarrow Y$に対して,以下の図式を可換にする射である.
\begin{center}
\begin{tikzcd}[sep=huge]
  X
  \ar[d, "g"']
  \ar[r, "0_{X,Y}"]
  \ar[dr, "0_{X,Z}"]
  & Y
  \ar[d, "f"]
  \\
  Y
  \ar[r, "0_{Y,Z}"']
  & Z
\end{tikzcd}
\end{center}
\end{defn}

有限双積からなる圏は,以下の構造と満たすべきいくつかの公理\cite{selinger09}を持つ対称モノイド圏と同値である.
\begin{itemize}
    \item (自然な射の族)複製$\Delta_A:A \rightarrow A \otimes A$
    \item (自然な射の族)削除$\Diamond_A:A \rightarrow I$
    \item (自然な射の族)合併$\nabla_A:A \otimes A \rightarrow A$
    \item (自然な射の族)追加$\square_A:I \rightarrow A$
\end{itemize}

有限双積を持つ圏は半加法圏\cite{LACK2012593}である.半加法圏$\mathcal{C}$の任意の射$f,g:A \rightarrow B$に対して射の加法
\begin{equation*}
    f+g = \nabla_B \circ (f \oplus g) \circ \Delta_A:A \rightarrow B
\end{equation*}
と定義できる\cite{maclane1950duality}.
よって有限双積を持つ圏は可換モノイドで豊穣化された圏($\mathbf{CMon}$--圏)である\cite[Def. 2.7.6]{giles_investigation_2014}.
(ここで$\mathbf{CMon}$は対象を可換モノイド,射を可換モノイド間の準同型射とする圏)

射の加法は任意の射$f,g,h:A \rightarrow B$について以下の等式を満たす.
\begin{align*}
  (f + g) + h &= f + (g + h) \\
  0_{A,B} + f &= f = f + 0_{A,B} \\
  f + g &= g + f
\end{align*}
また,射の合成は射の加法を保存し,任意の対象$X,Y$と任意の射$f,g:A\rightarrow B, e:X \rightarrow A, h:B\rightarrow Y$について,以下の等式を満たす.
\begin{align*}
  (f+g) \circ e &= f \circ e + g \circ e \\
  h \circ (f+g) &= h \circ f + h \circ g \\
  f \circ 0_{X,A} &= 0_{X,B} \\
  0_{B,Y} \circ f &= 0_{A,Y}
\end{align*}

% \begin{cor}
% 半加法圏間の関手$\mathbf{F}:\mathcal{C} \rightarrow \mathcal{D}$が有限双積を保存するとき,すなわち,
% 零対象に対して,$\mathbf{F}(O_\mathcal{C}) \cong O_\mathcal{D}$であり,
% 任意の対象$A,B \in \Obj(\mathcal{C})$に対して同型射$\mathbf{F}(A\oplus B) \cong \mathbf{F}(A)\oplus\mathbf{F}(B)$が存在するとき,かつその時に限り,$\mathbf{F}$を加法的関手(additive functor)と呼ぶ.
% \end{cor}
% \begin{proof}
%   $\Rightarrow$有限双積が保存されるとき,射の加法は
%   {\bf ToDo}
% \end{proof}

また,モノイダル圏$\mathcal{C}$について有限双積が保存されるとき,モノイダル積は射の加法を保存し,
任意の対象$X,Y$と任意の射$f,g:A \rightarrow B, e,h:C \rightarrow D$に対して,以下の等式を満たす.
\begin{align*}
  (f+g) \otimes h &= f \otimes h + g \otimes h \\
  f \otimes(e+h) &= f \otimes e + f \otimes h \\
  0_{X,Y} \otimes f &= 0_{X,Y}
\end{align*}

可換モノイドが冪等率$f+f=f$も満たす,すなわち射の加法が交わり(join)になっており,さらに完備上半束(sup--lattice),つまりホム対象の任意の部分集合について交わりを持つとき,
$\mathbf{CMon}$--圏はむしろ$\mathbf{Sup}$--圏である.

\begin{defn}[$\mathbf{Sup}$--Category($\mathbf{Sup}$--圏)]\cite[Def. 1.3.(i)]{pitts_applications_1988}
$\mathbf{Sup}$は対象が完備上半束,射が完備上半束を保存する写像からなる圏である.
$\mathbf{Sup}$--圏$\mathcal{C}$のホム対象$\mathcal{C}(A,B)$の任意の部分集合$S$に対して,$\mathcal{C}(A,B)$上の半順序に関する上限$\bigvee_{s\in S}s$を持つ.
射の合成は上限を保存し,任意の射$f:X \rightarrow A,g: B \rightarrow Y$について,以下の等式を満たす.
\begin{align*}
  \left(\bigvee_{s\in S}s\right) \circ f &= \bigvee_{s\in S}\left(s \circ f\right) \\
  g \circ \left(\bigvee_{s\in S}s\right) &= \bigvee_{s\in S}\left(g \circ s\right)
\end{align*}
\end{defn}

また,射の加法と同様に,モノイダル圏でもある$\mathbf{Sup}$--圏$\mathcal{C}$について,モノイダル積が上限を保存するとき,以下の等式を満たす.
\begin{align*}
  \left(\bigvee_{s\in S}s\right) \otimes f &= \bigvee_{s\in S}\left(s \otimes f\right) \\
  g \otimes \left(\bigvee_{s\in S}s\right) &= \bigvee_{s\in S}\left(g \otimes s\right)
\end{align*}

\begin{defn}[$\mathbf{Sup}$--Functor($\mathbf{Sup}$--関手)]\cite[Def. 1.3.(ii)]{pitts_applications_1988}
  $\mathbf{Sup}$--圏間の関手$\mathbf{F}:\mathcal{C}\rightarrow\mathcal{D}$が射の上限を保存するとき,$\mathbf{Sup}$--関手と呼ぶ.
  \begin{equation*}
    \mathbf{F}\left(\bigvee_{s\in S}s\right) = \bigvee_{s\in S}\mathbf{F}(s)
  \end{equation*}
\end{defn}

$\mathbf{Sup}$--圏はクオンタロイド(Quantaloid)という名前でも知られており\cite{stubbe_introduction_2014},multi-valuedな論理やファジィ論理などに応用されている.

\subsection{制限圏}
\begin{defn}[Restriction Category(制限圏)]\cite[Def. 2.1.1.]{cockett_restriction_2002}
圏$\mathcal{C}$について,$\mathcal{C}$の任意の射$f:A \rightarrow B$に対して射$\overline{f}:A \rightarrow A$が割り当てられ,
次の4つの条件を満たすとき,圏$\mathcal{C}$は制限構造を持ち,圏$\mathcal{C}$を制限圏と呼ぶ.
\begin{itemize}
  \item 任意の射$f:A \rightarrow B$について,$f \circ \overline{f} = f$
  \item 任意の射$f:A \rightarrow B$と$g:A \rightarrow C$について,$\overline{f}\circ\overline{g}=\overline{g}\circ\overline{f}$
  \item 任意の射$f:A \rightarrow B$と$g:A \rightarrow C$について,$\overline{g\circ\overline{f}}=\overline{g}\circ\overline{f}$
  \item 任意の射$f:A \rightarrow B$と$g:B \rightarrow C$について,$\overline{g}\circ{}f=f\circ\overline{g\circ{}f}$
\end{itemize}
ここで,射$\overline{f}$は$f$のrestriction idempotent(制限べき等)と呼ばれる.
\end{defn}

制限圏の射$f: A \rightarrow B$は$\overline{f} = \id_A$であるときtotalと呼ぶ.

制限圏は半順序集合$\mathrm{poset}$で豊穣化された圏となる\cite[p.237]{cockett_restriction_2002}\cite[Lemma 1.6.3]{guo_products_2012}.
射の半順序関係は次の形で与えられる.
\begin{equation*}
  f \leq g \Leftrightarrow f = g \circ \overline{f}
\end{equation*}

\begin{defn}[Restriction Functor(制限関手)]\cite[1.6.3 p.22]{guo_products_2012}
  制限圏間の関手$\mathbf{F}:\mathcal{C}\rightarrow\mathcal{D}$が制限構造を保存するとき,$\mathbf{F}$を制限関手と呼ぶ.
  \begin{equation*}
    \overline{\mathbf{F}(f)} = \mathbf{F(\overline{f})}
  \end{equation*}
\end{defn}

制限圏において,同型射を弱めた射を考える.
制限圏$\mathcal{C}$の射$f: A \rightarrow B$に対して$f^\circ \circ f = \overline{f}$と$f \circ f^\circ = \overline{f^\circ}$を満たす唯一の射を制限同型射(restricted isomorphism)あるいは部分同型射(partial isomorphism)と呼ぶ.

\begin{defn}[Inverse Category(逆圏)]\cite[section 2.3.2]{cockett_restriction_2002}\cite[Def. 3]{kaarsgaard_join_2017}
制限圏$\mathcal{C}$について,全ての射$f:A \rightarrow B$が部分同型射$f^\circ$を持つとき,$\mathcal{C}$を逆圏(Inverse Category)と呼ぶ.
\end{defn}

\begin{defn}[Join Restriction Category(結び制限圏)]{\bf 要調査}

  結び制限圏はposetというよりは完備上半束(sup--lattice)で豊穣化された圏
\end{defn}

\begin{defn}[Join Inverse Category(結び逆圏)]{\bf 要調査}
\end{defn}

Gilesの論文のDef. 10.3.1.で射の加法とモノイダル積,射の合成の組み合わせについて定義している(お互いを保存する形)


\subsection{その他構造}
\begin{defn}[Rig Category(半環圏)\cite{laplaza72}]
半環圏$\mathcal{C}$は以下の構造を備える.
\begin{itemize}
    \item (加法としての)対称モノイド構造$(\mathcal{C},\oplus,O)$
    \item (乗法としての)モノイド構造$(\mathcal{C},\otimes,I)$
    \item (自然同型)左分配子$\delta_l:A \otimes (B \oplus C) \cong (A \otimes B) \oplus (A \otimes C)$
    \item (自然同型)右分配子$\delta_r:(A \oplus B) \otimes C \cong (A \otimes C) \oplus (B \otimes C)$
    \item (自然同型)左吸収子$\kappa_l:A \otimes O \cong O$
    \item (自然同型)右吸収子$\kappa_r:O \otimes A \cong O$
    \item 諸々のコヒーレンス公理\cite{laplaza72}
\end{itemize}
\end{defn}

\begin{defn}[Trace(トレース)]{\bf 要調査}
\end{defn}

% \begin{defn}[Algebraically Compact Category(代数的コンパクト圏)]{\bf 要調査}

% \end{defn}

% 参考文献
\bibliography{reference}
\bibliographystyle{junsrt}

\end{document}